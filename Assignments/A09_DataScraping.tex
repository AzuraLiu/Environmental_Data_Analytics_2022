% Options for packages loaded elsewhere
\PassOptionsToPackage{unicode}{hyperref}
\PassOptionsToPackage{hyphens}{url}
%
\documentclass[
]{article}
\usepackage{amsmath,amssymb}
\usepackage{lmodern}
\usepackage{iftex}
\ifPDFTeX
  \usepackage[T1]{fontenc}
  \usepackage[utf8]{inputenc}
  \usepackage{textcomp} % provide euro and other symbols
\else % if luatex or xetex
  \usepackage{unicode-math}
  \defaultfontfeatures{Scale=MatchLowercase}
  \defaultfontfeatures[\rmfamily]{Ligatures=TeX,Scale=1}
\fi
% Use upquote if available, for straight quotes in verbatim environments
\IfFileExists{upquote.sty}{\usepackage{upquote}}{}
\IfFileExists{microtype.sty}{% use microtype if available
  \usepackage[]{microtype}
  \UseMicrotypeSet[protrusion]{basicmath} % disable protrusion for tt fonts
}{}
\makeatletter
\@ifundefined{KOMAClassName}{% if non-KOMA class
  \IfFileExists{parskip.sty}{%
    \usepackage{parskip}
  }{% else
    \setlength{\parindent}{0pt}
    \setlength{\parskip}{6pt plus 2pt minus 1pt}}
}{% if KOMA class
  \KOMAoptions{parskip=half}}
\makeatother
\usepackage{xcolor}
\IfFileExists{xurl.sty}{\usepackage{xurl}}{} % add URL line breaks if available
\IfFileExists{bookmark.sty}{\usepackage{bookmark}}{\usepackage{hyperref}}
\hypersetup{
  pdftitle={Assignment 09: Data Scraping},
  pdfauthor={Azura Liu},
  hidelinks,
  pdfcreator={LaTeX via pandoc}}
\urlstyle{same} % disable monospaced font for URLs
\usepackage[margin=2.54cm]{geometry}
\usepackage{color}
\usepackage{fancyvrb}
\newcommand{\VerbBar}{|}
\newcommand{\VERB}{\Verb[commandchars=\\\{\}]}
\DefineVerbatimEnvironment{Highlighting}{Verbatim}{commandchars=\\\{\}}
% Add ',fontsize=\small' for more characters per line
\usepackage{framed}
\definecolor{shadecolor}{RGB}{248,248,248}
\newenvironment{Shaded}{\begin{snugshade}}{\end{snugshade}}
\newcommand{\AlertTok}[1]{\textcolor[rgb]{0.94,0.16,0.16}{#1}}
\newcommand{\AnnotationTok}[1]{\textcolor[rgb]{0.56,0.35,0.01}{\textbf{\textit{#1}}}}
\newcommand{\AttributeTok}[1]{\textcolor[rgb]{0.77,0.63,0.00}{#1}}
\newcommand{\BaseNTok}[1]{\textcolor[rgb]{0.00,0.00,0.81}{#1}}
\newcommand{\BuiltInTok}[1]{#1}
\newcommand{\CharTok}[1]{\textcolor[rgb]{0.31,0.60,0.02}{#1}}
\newcommand{\CommentTok}[1]{\textcolor[rgb]{0.56,0.35,0.01}{\textit{#1}}}
\newcommand{\CommentVarTok}[1]{\textcolor[rgb]{0.56,0.35,0.01}{\textbf{\textit{#1}}}}
\newcommand{\ConstantTok}[1]{\textcolor[rgb]{0.00,0.00,0.00}{#1}}
\newcommand{\ControlFlowTok}[1]{\textcolor[rgb]{0.13,0.29,0.53}{\textbf{#1}}}
\newcommand{\DataTypeTok}[1]{\textcolor[rgb]{0.13,0.29,0.53}{#1}}
\newcommand{\DecValTok}[1]{\textcolor[rgb]{0.00,0.00,0.81}{#1}}
\newcommand{\DocumentationTok}[1]{\textcolor[rgb]{0.56,0.35,0.01}{\textbf{\textit{#1}}}}
\newcommand{\ErrorTok}[1]{\textcolor[rgb]{0.64,0.00,0.00}{\textbf{#1}}}
\newcommand{\ExtensionTok}[1]{#1}
\newcommand{\FloatTok}[1]{\textcolor[rgb]{0.00,0.00,0.81}{#1}}
\newcommand{\FunctionTok}[1]{\textcolor[rgb]{0.00,0.00,0.00}{#1}}
\newcommand{\ImportTok}[1]{#1}
\newcommand{\InformationTok}[1]{\textcolor[rgb]{0.56,0.35,0.01}{\textbf{\textit{#1}}}}
\newcommand{\KeywordTok}[1]{\textcolor[rgb]{0.13,0.29,0.53}{\textbf{#1}}}
\newcommand{\NormalTok}[1]{#1}
\newcommand{\OperatorTok}[1]{\textcolor[rgb]{0.81,0.36,0.00}{\textbf{#1}}}
\newcommand{\OtherTok}[1]{\textcolor[rgb]{0.56,0.35,0.01}{#1}}
\newcommand{\PreprocessorTok}[1]{\textcolor[rgb]{0.56,0.35,0.01}{\textit{#1}}}
\newcommand{\RegionMarkerTok}[1]{#1}
\newcommand{\SpecialCharTok}[1]{\textcolor[rgb]{0.00,0.00,0.00}{#1}}
\newcommand{\SpecialStringTok}[1]{\textcolor[rgb]{0.31,0.60,0.02}{#1}}
\newcommand{\StringTok}[1]{\textcolor[rgb]{0.31,0.60,0.02}{#1}}
\newcommand{\VariableTok}[1]{\textcolor[rgb]{0.00,0.00,0.00}{#1}}
\newcommand{\VerbatimStringTok}[1]{\textcolor[rgb]{0.31,0.60,0.02}{#1}}
\newcommand{\WarningTok}[1]{\textcolor[rgb]{0.56,0.35,0.01}{\textbf{\textit{#1}}}}
\usepackage{graphicx}
\makeatletter
\def\maxwidth{\ifdim\Gin@nat@width>\linewidth\linewidth\else\Gin@nat@width\fi}
\def\maxheight{\ifdim\Gin@nat@height>\textheight\textheight\else\Gin@nat@height\fi}
\makeatother
% Scale images if necessary, so that they will not overflow the page
% margins by default, and it is still possible to overwrite the defaults
% using explicit options in \includegraphics[width, height, ...]{}
\setkeys{Gin}{width=\maxwidth,height=\maxheight,keepaspectratio}
% Set default figure placement to htbp
\makeatletter
\def\fps@figure{htbp}
\makeatother
\setlength{\emergencystretch}{3em} % prevent overfull lines
\providecommand{\tightlist}{%
  \setlength{\itemsep}{0pt}\setlength{\parskip}{0pt}}
\setcounter{secnumdepth}{-\maxdimen} % remove section numbering
\ifLuaTeX
  \usepackage{selnolig}  % disable illegal ligatures
\fi

\title{Assignment 09: Data Scraping}
\author{Azura Liu}
\date{}

\begin{document}
\maketitle

\hypertarget{total-points}{%
\section{Total points:}\label{total-points}}

\hypertarget{overview}{%
\subsection{OVERVIEW}\label{overview}}

This exercise accompanies the lessons in Environmental Data Analytics on
data scraping.

\hypertarget{directions}{%
\subsection{Directions}\label{directions}}

\begin{enumerate}
\def\labelenumi{\arabic{enumi}.}
\tightlist
\item
  Change ``Student Name'' on line 3 (above) with your name.
\item
  Work through the steps, \textbf{creating code and output} that fulfill
  each instruction.
\item
  Be sure to \textbf{answer the questions} in this assignment document.
\item
  When you have completed the assignment, \textbf{Knit} the text and
  code into a single PDF file.
\item
  After Knitting, submit the completed exercise (PDF file) to the
  dropbox in Sakai. Add your last name into the file name (e.g.,
  ``Fay\_09\_Data\_Scraping.Rmd'') prior to submission.
\end{enumerate}

\hypertarget{set-up}{%
\subsection{Set up}\label{set-up}}

\begin{enumerate}
\def\labelenumi{\arabic{enumi}.}
\tightlist
\item
  Set up your session:
\end{enumerate}

\begin{itemize}
\tightlist
\item
  Check your working directory
\item
  Load the packages \texttt{tidyverse}, \texttt{rvest}, and any others
  you end up using.
\item
  Set your ggplot theme
\end{itemize}

\begin{Shaded}
\begin{Highlighting}[]
\CommentTok{\#1}
\FunctionTok{getwd}\NormalTok{()}
\end{Highlighting}
\end{Shaded}

\begin{verbatim}
## [1] "C:/Users/yliua/Desktop/Environmental_Data_Analytics_2022/Assignments"
\end{verbatim}

\begin{Shaded}
\begin{Highlighting}[]
\FunctionTok{library}\NormalTok{(tidyverse)}
\FunctionTok{library}\NormalTok{(rvest)}
\FunctionTok{library}\NormalTok{(lubridate)}

\NormalTok{mytheme }\OtherTok{\textless{}{-}} \FunctionTok{theme\_classic}\NormalTok{() }\SpecialCharTok{+}
  \FunctionTok{theme}\NormalTok{(}\AttributeTok{axis.text =} \FunctionTok{element\_text}\NormalTok{(}\AttributeTok{color =} \StringTok{"black"}\NormalTok{), }
        \AttributeTok{legend.position =} \StringTok{"top"}\NormalTok{)}
\FunctionTok{theme\_set}\NormalTok{(mytheme)}
\end{Highlighting}
\end{Shaded}

\begin{enumerate}
\def\labelenumi{\arabic{enumi}.}
\setcounter{enumi}{1}
\tightlist
\item
  We will be scraping data from the NC DEQs Local Water Supply Planning
  website, specifically the Durham's 2020 Municipal Local Water Supply
  Plan (LWSP):
\end{enumerate}

\begin{itemize}
\tightlist
\item
  Navigate to \url{https://www.ncwater.org/WUDC/app/LWSP/search.php}
\item
  Change the date from 2020 to 2020 in the upper right corner.
\item
  Scroll down and select the LWSP link next to Durham Municipality.
\item
  Note the web address:
  \url{https://www.ncwater.org/WUDC/app/LWSP/report.php?pwsid=03-32-010\&year=2020}
\end{itemize}

Indicate this website as the as the URL to be scraped. (In other words,
read the contents into an \texttt{rvest} webpage object.)

\begin{Shaded}
\begin{Highlighting}[]
\CommentTok{\#2}
\NormalTok{web}\OtherTok{\textless{}{-}}\FunctionTok{read\_html}\NormalTok{(}\StringTok{"https://www.ncwater.org/WUDC/app/LWSP/report.php?pwsid=03{-}32{-}010\&year=2020"}\NormalTok{)}
\end{Highlighting}
\end{Shaded}

\begin{enumerate}
\def\labelenumi{\arabic{enumi}.}
\setcounter{enumi}{2}
\tightlist
\item
  The data we want to collect are listed below:
\end{enumerate}

\begin{itemize}
\item
  From the ``1. System Information'' section:
\item
  Water system name
\item
  PSWID
\item
  Ownership
\item
  From the ``3. Water Supply Sources'' section:
\item
  max Daily Use (MGD) - for each month
\end{itemize}

In the code chunk below scrape these values, assigning them to three
separate variables.

\begin{quote}
HINT: The first value should be ``Durham'', the second ``03-32-010'',
the third ``Municipality'', and the last should be a vector of 12
numeric values, with the first value being 36.0100.
\end{quote}

\begin{Shaded}
\begin{Highlighting}[]
\CommentTok{\#3}
\NormalTok{water.system.name }\OtherTok{\textless{}{-}}\NormalTok{ web }\SpecialCharTok{\%\textgreater{}\%} 
  \FunctionTok{html\_nodes}\NormalTok{(}\StringTok{"div+ table tr:nth{-}child(1) td:nth{-}child(2)"}\NormalTok{) }\SpecialCharTok{\%\textgreater{}\%} 
  \FunctionTok{html\_text}\NormalTok{()}

\NormalTok{pswid }\OtherTok{\textless{}{-}}\NormalTok{  web }\SpecialCharTok{\%\textgreater{}\%} 
  \FunctionTok{html\_nodes}\NormalTok{(}\StringTok{"td tr:nth{-}child(1) td:nth{-}child(5)"}\NormalTok{) }\SpecialCharTok{\%\textgreater{}\%} 
  \FunctionTok{html\_text}\NormalTok{()}

\NormalTok{ownership }\OtherTok{\textless{}{-}}\NormalTok{  web }\SpecialCharTok{\%\textgreater{}\%} 
  \FunctionTok{html\_nodes}\NormalTok{(}\StringTok{"div+ table tr:nth{-}child(2) td:nth{-}child(4)"}\NormalTok{) }\SpecialCharTok{\%\textgreater{}\%} 
  \FunctionTok{html\_text}\NormalTok{()}

\NormalTok{max.withdrawals.mgd }\OtherTok{\textless{}{-}}\NormalTok{  web }\SpecialCharTok{\%\textgreater{}\%} 
  \FunctionTok{html\_nodes}\NormalTok{(}\StringTok{"th\textasciitilde{} td+ td"}\NormalTok{) }\SpecialCharTok{\%\textgreater{}\%} 
  \FunctionTok{html\_text}\NormalTok{()}
\end{Highlighting}
\end{Shaded}

\begin{enumerate}
\def\labelenumi{\arabic{enumi}.}
\setcounter{enumi}{3}
\tightlist
\item
  Convert your scraped data into a dataframe. This dataframe should have
  a column for each of the 4 variables scraped and a row for the month
  corresponding to the withdrawal data. Also add a Date column that
  includes your month and year in data format. (Feel free to add a Year
  column too, if you wish.)
\end{enumerate}

\begin{quote}
TIP: Use \texttt{rep()} to repeat a value when creating a dataframe.
\end{quote}

\begin{quote}
NOTE: It's likely you won't be able to scrape the monthly widthrawal
data in order. You can overcome this by creating a month column in the
same order the data are scraped: Jan, May, Sept, Feb, etc\ldots{}
\end{quote}

\begin{enumerate}
\def\labelenumi{\arabic{enumi}.}
\setcounter{enumi}{4}
\tightlist
\item
  Plot the max daily withdrawals across the months for 2020
\end{enumerate}

\begin{Shaded}
\begin{Highlighting}[]
\CommentTok{\#4}
\NormalTok{df\_withdrawals }\OtherTok{\textless{}{-}} \FunctionTok{data.frame}\NormalTok{(}\StringTok{"Month"} \OtherTok{=} \FunctionTok{c}\NormalTok{(}\StringTok{"Jan"}\NormalTok{,}\StringTok{"May"}\NormalTok{,}\StringTok{"Sep"}\NormalTok{,}\StringTok{"Feb"}\NormalTok{,}\StringTok{"Jun"}\NormalTok{, }\StringTok{"Oct"}\NormalTok{, }\StringTok{"Mar"}\NormalTok{,}\StringTok{"Jul"}\NormalTok{,}\StringTok{"Nov"}\NormalTok{,}\StringTok{"Apr"}\NormalTok{,}\StringTok{"Aug"}\NormalTok{,}\StringTok{"Dec"}\NormalTok{),}
                             \StringTok{"Year"} \OtherTok{=} \FunctionTok{rep}\NormalTok{(}\DecValTok{2020}\NormalTok{,}\DecValTok{12}\NormalTok{),}
                             \StringTok{"water.system.name"} \OtherTok{=}\NormalTok{ water.system.name,}
                             \StringTok{"pswid"}\OtherTok{=}\NormalTok{ pswid,}
                             \StringTok{"ownership"}\OtherTok{=}\NormalTok{ownership,}
                             \StringTok{"max.withdrawals.mgd"} \OtherTok{=} \FunctionTok{as.numeric}\NormalTok{(max.withdrawals.mgd))}\SpecialCharTok{\%\textgreater{}\%}
  \FunctionTok{mutate}\NormalTok{(}\AttributeTok{Date =}\FunctionTok{my}\NormalTok{(}\FunctionTok{paste}\NormalTok{(Month,Year)))}


\CommentTok{\#5}
\FunctionTok{ggplot}\NormalTok{(df\_withdrawals,}\FunctionTok{aes}\NormalTok{(}\AttributeTok{x=}\NormalTok{ Date,}\AttributeTok{y=}\NormalTok{max.withdrawals.mgd)) }\SpecialCharTok{+} 
  \FunctionTok{geom\_line}\NormalTok{() }\SpecialCharTok{+} 
  \FunctionTok{labs}\NormalTok{(}\AttributeTok{title =} \StringTok{"Max daily withdrawals across the months for 2020"}\NormalTok{,}
       \AttributeTok{y=}\StringTok{"Withdrawal (mgd)"}\NormalTok{,}
       \AttributeTok{x=}\StringTok{"Date"}\NormalTok{)}
\end{Highlighting}
\end{Shaded}

\includegraphics{A09_DataScraping_files/figure-latex/create.a.dataframe.from.scraped.data-1.pdf}

\begin{enumerate}
\def\labelenumi{\arabic{enumi}.}
\setcounter{enumi}{5}
\tightlist
\item
  Note that the PWSID and the year appear in the web address for the
  page we scraped. Construct a function using your code above that can
  scrape data for any PWSID and year for which the NC DEQ has data.
  \textbf{Be sure to modify the code to reflect the year and site
  scraped}.
\end{enumerate}

\begin{Shaded}
\begin{Highlighting}[]
\CommentTok{\#6.}
\NormalTok{scrape.it }\OtherTok{\textless{}{-}} \ControlFlowTok{function}\NormalTok{(the\_year, the\_pwsid)\{}
\NormalTok{  the\_scrape\_web}\OtherTok{\textless{}{-}}\FunctionTok{read\_html}\NormalTok{(}\FunctionTok{paste0}\NormalTok{(}\StringTok{"https://www.ncwater.org/WUDC/app/LWSP/report.php?"}\NormalTok{, }\StringTok{"pwsid="}\NormalTok{, the\_pwsid, }\StringTok{"\&year="}\NormalTok{, the\_year))}

\NormalTok{  water.system.name.tag }\OtherTok{\textless{}{-}}\StringTok{"div+ table tr:nth{-}child(1) td:nth{-}child(2)"}
\NormalTok{  pswid.tag }\OtherTok{\textless{}{-}}\StringTok{"td tr:nth{-}child(1) td:nth{-}child(5)"}
\NormalTok{  ownership.tag }\OtherTok{\textless{}{-}}\StringTok{"div+ table tr:nth{-}child(2) td:nth{-}child(4)"}
\NormalTok{  max.withdrawals.mgd.tag }\OtherTok{\textless{}{-}}  \StringTok{"th\textasciitilde{} td+ td"}

\NormalTok{  the.water.system.name }\OtherTok{\textless{}{-}}\NormalTok{ the\_scrape\_web }\SpecialCharTok{\%\textgreater{}\%} \FunctionTok{html\_nodes}\NormalTok{(water.system.name.tag) }\SpecialCharTok{\%\textgreater{}\%} \FunctionTok{html\_text}\NormalTok{()}
\NormalTok{  the.pswid }\OtherTok{\textless{}{-}}\NormalTok{ the\_scrape\_web }\SpecialCharTok{\%\textgreater{}\%}   \FunctionTok{html\_nodes}\NormalTok{(pswid.tag) }\SpecialCharTok{\%\textgreater{}\%}  \FunctionTok{html\_text}\NormalTok{()}
\NormalTok{  the.ownership }\OtherTok{\textless{}{-}}\NormalTok{ the\_scrape\_web }\SpecialCharTok{\%\textgreater{}\%} \FunctionTok{html\_nodes}\NormalTok{(ownership.tag) }\SpecialCharTok{\%\textgreater{}\%} \FunctionTok{html\_text}\NormalTok{()}
\NormalTok{  the.max.withdrawals.mgd }\OtherTok{\textless{}{-}}\NormalTok{ the\_scrape\_web }\SpecialCharTok{\%\textgreater{}\%} \FunctionTok{html\_nodes}\NormalTok{(max.withdrawals.mgd.tag) }\SpecialCharTok{\%\textgreater{}\%} \FunctionTok{html\_text}\NormalTok{()}

\NormalTok{  df\_custom }\OtherTok{\textless{}{-}} \FunctionTok{data.frame}\NormalTok{(}\StringTok{"Month"} \OtherTok{=} \FunctionTok{c}\NormalTok{(}\StringTok{"Jan"}\NormalTok{,}\StringTok{"May"}\NormalTok{,}\StringTok{"Sep"}\NormalTok{,}\StringTok{"Feb"}\NormalTok{,}\StringTok{"Jun"}\NormalTok{, }\StringTok{"Oct"}\NormalTok{, }\StringTok{"Mar"}\NormalTok{,}\StringTok{"Jul"}\NormalTok{,}\StringTok{"Nov"}\NormalTok{,}\StringTok{"Apr"}\NormalTok{,}\StringTok{"Aug"}\NormalTok{,}\StringTok{"Dec"}\NormalTok{),}
                        \StringTok{"Year"} \OtherTok{=} \FunctionTok{rep}\NormalTok{(the\_year,}\DecValTok{12}\NormalTok{),}
                        \StringTok{"max.withdrawals.mgd"} \OtherTok{=} \FunctionTok{as.numeric}\NormalTok{(the.max.withdrawals.mgd))}\SpecialCharTok{\%\textgreater{}\%}
    \FunctionTok{mutate}\NormalTok{(}\AttributeTok{the.water.system.name =} \SpecialCharTok{!!}\NormalTok{the.water.system.name,}
         \AttributeTok{the.pswid =} \SpecialCharTok{!!}\NormalTok{the.pswid,}
         \AttributeTok{the.ownership =} \SpecialCharTok{!!}\NormalTok{the.ownership,}
         \AttributeTok{Date =} \FunctionTok{my}\NormalTok{(}\FunctionTok{paste}\NormalTok{(Month,}\StringTok{"{-}"}\NormalTok{,the\_year)))}
  \FunctionTok{return}\NormalTok{(df\_custom)}
\NormalTok{\}}
\end{Highlighting}
\end{Shaded}

\begin{enumerate}
\def\labelenumi{\arabic{enumi}.}
\setcounter{enumi}{6}
\tightlist
\item
  Use the function above to extract and plot max daily withdrawals for
  Durham (PWSID=`03-32-010') for each month in 2015
\end{enumerate}

\begin{Shaded}
\begin{Highlighting}[]
\CommentTok{\#7}
\NormalTok{df\_2015}\OtherTok{\textless{}{-}}\FunctionTok{scrape.it}\NormalTok{(}\DecValTok{2015}\NormalTok{,}\StringTok{\textquotesingle{}03{-}32{-}010\textquotesingle{}}\NormalTok{)}
\FunctionTok{view}\NormalTok{(df\_2015)}

\FunctionTok{ggplot}\NormalTok{(df\_2015,}\FunctionTok{aes}\NormalTok{(}\AttributeTok{x=}\NormalTok{ Date,}\AttributeTok{y=}\NormalTok{max.withdrawals.mgd)) }\SpecialCharTok{+} 
  \FunctionTok{geom\_line}\NormalTok{() }\SpecialCharTok{+} 
  \FunctionTok{labs}\NormalTok{(}\AttributeTok{title =} \StringTok{"Max daily withdrawals across the months for 2020"}\NormalTok{,}
       \AttributeTok{y=}\StringTok{"Withdrawal (mgd)"}\NormalTok{,}
       \AttributeTok{x=}\StringTok{"Date"}\NormalTok{)}
\end{Highlighting}
\end{Shaded}

\includegraphics{A09_DataScraping_files/figure-latex/fetch.and.plot.Durham.2015.data-1.pdf}

\begin{enumerate}
\def\labelenumi{\arabic{enumi}.}
\setcounter{enumi}{7}
\tightlist
\item
  Use the function above to extract data for Asheville (PWSID =
  01-11-010) in 2015. Combine this data with the Durham data collected
  above and create a plot that compares the Asheville to Durham's water
  withdrawals.
\end{enumerate}

\begin{Shaded}
\begin{Highlighting}[]
\CommentTok{\#8}
\NormalTok{df\_Ash}\OtherTok{\textless{}{-}}\FunctionTok{scrape.it}\NormalTok{(}\DecValTok{2015}\NormalTok{,}\StringTok{\textquotesingle{}01{-}11{-}010\textquotesingle{}}\NormalTok{)}
\FunctionTok{view}\NormalTok{(df\_Ash)}

\FunctionTok{ggplot}\NormalTok{(df\_Ash,}\FunctionTok{aes}\NormalTok{(Date)) }\SpecialCharTok{+} 
  \FunctionTok{geom\_line}\NormalTok{(}\FunctionTok{aes}\NormalTok{(}\AttributeTok{y=}\NormalTok{max.withdrawals.mgd,}\AttributeTok{color =} \StringTok{"Ashville"}\NormalTok{)) }\SpecialCharTok{+} 
  \FunctionTok{geom\_line}\NormalTok{(}\FunctionTok{aes}\NormalTok{(}\AttributeTok{y=}\NormalTok{df\_2015}\SpecialCharTok{$}\NormalTok{max.withdrawals.mgd, }\AttributeTok{color =} \StringTok{"Durham"}\NormalTok{))}\SpecialCharTok{+}
  \FunctionTok{labs}\NormalTok{(}\AttributeTok{title =} \StringTok{"Max daily withdrawals between Durham and Ashville in 2015"}\NormalTok{,}
       \AttributeTok{y=}\StringTok{"Withdrawal (mgd)"}\NormalTok{,}
       \AttributeTok{x=}\StringTok{"Date"}\NormalTok{)}
\end{Highlighting}
\end{Shaded}

\includegraphics{A09_DataScraping_files/figure-latex/fetch.and.plot.Durham.and.Ashville.2015.data-1.pdf}

\begin{enumerate}
\def\labelenumi{\arabic{enumi}.}
\setcounter{enumi}{8}
\tightlist
\item
  Use the code \& function you created above to plot Asheville's max
  daily withdrawal by months for the years 2010 thru 2020.Add a smoothed
  line to the plot.
\end{enumerate}

\begin{Shaded}
\begin{Highlighting}[]
\CommentTok{\#9}
\NormalTok{the\_years }\OtherTok{=} \FunctionTok{rep}\NormalTok{(}\DecValTok{2010}\SpecialCharTok{:}\DecValTok{2020}\NormalTok{)}
\NormalTok{Ashville }\OtherTok{=} \StringTok{\textquotesingle{}01{-}11{-}010\textquotesingle{}}

\NormalTok{the\_dfs }\OtherTok{\textless{}{-}} \FunctionTok{lapply}\NormalTok{(}\AttributeTok{X =}\NormalTok{ the\_years,}
                  \AttributeTok{FUN =}\NormalTok{ scrape.it,}
                  \AttributeTok{the\_pwsid=}\NormalTok{Ashville)}

\NormalTok{the\_dfs }\OtherTok{\textless{}{-}} \FunctionTok{map}\NormalTok{(the\_years,scrape.it, }\AttributeTok{the\_pwsid=}\NormalTok{Ashville)}

\NormalTok{the\_df }\OtherTok{\textless{}{-}} \FunctionTok{bind\_rows}\NormalTok{(the\_dfs)}

\FunctionTok{ggplot}\NormalTok{(the\_df,}\FunctionTok{aes}\NormalTok{(}\AttributeTok{x=}\NormalTok{Date,}\AttributeTok{y=}\NormalTok{max.withdrawals.mgd)) }\SpecialCharTok{+} 
  \FunctionTok{geom\_line}\NormalTok{() }\SpecialCharTok{+} 
  \FunctionTok{geom\_smooth}\NormalTok{(}\AttributeTok{method=}\StringTok{"loess"}\NormalTok{,}\AttributeTok{se=}\ConstantTok{FALSE}\NormalTok{) }\SpecialCharTok{+}
  \FunctionTok{labs}\NormalTok{(}\AttributeTok{title =} \FunctionTok{paste}\NormalTok{(}\StringTok{"2010{-}2020 Maximum Water Withdraw in Ashvielle"}\NormalTok{),}
       \AttributeTok{y=}\StringTok{"Max Withdrawal (mgd)"}\NormalTok{,}
       \AttributeTok{x=}\StringTok{"Date"}\NormalTok{)}
\end{Highlighting}
\end{Shaded}

\begin{verbatim}
## `geom_smooth()` using formula 'y ~ x'
\end{verbatim}

\includegraphics{A09_DataScraping_files/figure-latex/unnamed-chunk-2-1.pdf}

\begin{quote}
Question: Just by looking at the plot (i.e.~not running statistics),
does Asheville have a trend in water usage over time? Yes, the water
usage appears to increase over time.
\end{quote}

\end{document}
